\section{Zielsetzung}
Ziel dieses Versuches ist es, die Temperaturabhängigkeit der Molwärme von Kupfer
zu messen und die Debye-Temperatur zu bestimmen.
\section{Theorie}
Die Molwärme beschreibt die Wärmemenge, die benötigt wird um ein Mol eines
Stoffes um einen Kelvin zu erwärmen. Für die Beschreibung der
Temperaturabhängigkeit der Molwärme für kristalline Festkörper
werden drei Modelle herangezogen.

\subsection{Das klassische Modell}
Das Äquipartitionsprinzip aus der klassischen Mechanik besagt, dass sich die
Wärmeenergie, die einem Körper zugeführt wird, gleichmäßig auf alle
Bewegungsfreiheitsgrade der Atome verteilt. Für die mittlere Energie gilt dann
\begin{equation}
  \langle E_{\text{kin}} \rangle = \frac{\text{f}}{2} \text{k} T \, .
\end{equation}
Die verwendeten Parameter beschreiben dabei die Bolzmannsche Konstante
$\text{k}$, die Anzahl der Freiheitsgrade $\text{f}$ und die Temperatur $T$. Bei
harmonisch schwingenden Atomen gilt, dass die mittlere potentielle Energie
der mittleren kinetischen Energie entspricht. Ein Atom kann sich in einem
Atom in drei senkrecht aufeinander stehende Bewegungsrichtungen bewegen. Es
besitzt somit drei Freiheitsgrade. Dadurch folgt für die mittlere Energie
\begin{align*}
  \langle E \rangle &= 2 \cdot \frac{3}{2} \text{k} T \\
                    &= 3 \text{k} T  \, .
\end{align*}
Nach der Umrechnung für ein Mol in einem Kristall gilt dann für die Energie
\begin{equation}
  E = 3 \text{R} T \, .
\end{equation}
Für die spezifische Molwärme bei konstantem Volumen gilt dann
\begin{equation}
  \left(\frac{\partial E}{\partial T}\right)_{\!\! V} = 3 \text{R} \, .
\end{equation}
Dem klassischen Model nach gibt es bei der Molwärme somit keine
materialabhängigen Eigenschaften. Auch die Temperatur wird vernachlässigt.
Es zeigt sich jedoch, dass der Wert $3 \text{R}$ nur asymptotisch bei
ausreichend hohen Temperaturen erreicht werden kann.

\subsection{Das Einstein-Modell}
Das klassische Modell vernachlässigt, dass die Atome auf den Gitterplätzen
mit verschiedenen Kreisfrequenzen oszillieren. Das Einstein-Modell nähert,
dass alle Atome mit der gleichen Kreisfrequenz $\omega$ schwingen. Die Atome
können nur diskrete Energien mit dem Werten $n \hbar \omega$ aufnehmen und
abgeben.
