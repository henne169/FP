\section{Zielsetzung}
Ziel dieses Versuches ist es, die Temperaturabhängigkeit der Molwärme von Kupfer
zu messen und die Debye-Temperatur zu bestimmen.
\section{Theorie}
Die Molwärme beschreibt die Wärmemenge, die benötigt wird um ein Mol eines
Stoffes um einen Kelvin zu erwärmen. Für die Beschreibung der
Temperaturabhängigkeit der Molwärme für kristalline Festkörper
werden drei Modelle herangezogen.

\subsection{Das klassische Modell}
Das Äquipartitionsprinzip aus der klassischen Mechanik besagt, dass sich die
Wärmeenergie, die einem Körper zugeführt wird, gleichmäßig auf alle
Bewegungsfreiheitsgrade der Atome verteilt. Für die mittlere Energie gilt dann
\begin{equation}
  \langle E_{\text{kin}} \rangle = \frac{\text{f}}{2} \text{k} T \, .
\end{equation}
Die verwendeten Parameter beschreiben dabei die Bolzmannsche Konstante
$\text{k}$, die Anzahl der Freiheitsgrade $\text{f}$ und die Temperatur $T$. Bei
harmonisch schwingenden Atomen gilt, dass die mittlere potentielle Energie
der mittleren kinetischen Energie entspricht. Ein Atom kann sich in einem
Atom in drei senkrecht aufeinander stehende Bewegungsrichtungen bewegen. Es
besitzt somit drei Freiheitsgrade. Dadurch folgt für die mittlere Energie
\begin{align*}
  \langle E \rangle &= 2 \cdot \frac{3}{2} \text{k} T \\
                    &= 3 \text{k} T  \, .
\end{align*}
Nach der Umrechnung für ein Mol in einem Kristall gilt dann für die Energie
\begin{equation}
  E = 3 \text{R} T \, .
\end{equation}
Für die spezifische Molwärme bei konstantem Volumen gilt dann
\begin{equation}
  \left(\frac{\partial E}{\partial T}\right)_{\!\! V} = 3 \text{R} \, .
\end{equation}
Dem klassischen Model nach gibt es bei der Molwärme somit keine
materialabhängigen Eigenschaften. Auch die Temperatur wird vernachlässigt.
Es zeigt sich jedoch, dass der Wert $3 \text{R}$ nur asymptotisch bei
ausreichend hohen Temperaturen erreicht werden kann.

\subsection{Das Einstein-Modell}
Das klassische Modell vernachlässigt, dass die Atome auf den Gitterplätzen
mit verschiedenen Kreisfrequenzen oszillieren. Das Einstein-Modell nähert,
dass alle Atome mit der gleichen Kreisfrequenz $\omega$ schwingen. Die Atome
können nur diskrete Energien mit dem Werten $n \hbar \omega$ aufnehmen und
abgeben. Zur Berechnung der mittlere Energie pro Oszillator, wird die
Boltzmann-Verteilung der Energieniveaus $n$ benötigt. Dadurch ergibt sich
für die mittlere Energie
\begin{align*}
  \frac{\hbar \omega}{\exp\left(\frac{\hbar \omega}{\text{k} T} \right) -1}
\end{align*}

Erneut wird der Term nach der Temperatur abgeleitet, wodurch sich eine Molwärme
von
\begin{equation}
  C_{\text{V}} = \frac{3 \text{R} \hbar^2 \omega^2}{\text{k}^2 T^2}
  \frac{\exp\left(\frac{\hbar \omega}{\text{k} T}\right)}{\left(\exp\left(\frac{\hbar \omega}{\text{k} T}\right) -1 \right)^2}
\end{equation}

Bei hohen Temperaturen nähert sich die Einstein-Funktion dem Wert $3 \text{R}$.
Zwar beschreibt das Einstein-Modell eine Abnahme der Molwärme bei niedrigeren
Temperaturen, jedoch gibt es durch die grobe Näherung der Kreisfrequenz eine
Abweichung zu den experimentell ermittelten Werten in Tieftemperaturbereich.

\subsection{Das Debye-Modell}
Das Deby-Modell geht nicht wie im Einstein-Modell von einer einheitlichen
Kreisfrequenz $\omega$ aus, sondern ordnet der oszillatorischen Bewegungen
der Atome auf den Gitterplätzen eine spektrale Frequenzverteilung $Z(\omega)$
zu. Da die Funktion $Z(\omega)$ auf Grund des stark elastischen Verhaltens
von Kristallen sehr kompliziert werden kann, wird genähert, dass die Frequenz
und die Ausbreitungsrichtung einer Welle im Kristall keinen Einfluss auf
ihre Phasengeschwindigkeit hat. Dadurch ergibt sich für die Funktion $Z(\omega)$
unter der Voraussetzung, dass Longitudinal- und Transversalwellen verschiedene
Phasengeschwindigkeiten haben:
\begin{equation}
  Z(\omega) \symup{d} \omega = \frac{L^3 \omega^2}{2 \pi}
  \left(\frac{1}{v_l^3} + \frac{1}{v_{tr}^3} \right) \symup{d}\omega
\end{equation}
Die Debye-Frequenz $\omega_D$, also die obere Grenzfrequenz, existiert, da ein
Kristall mit endlichen Dimensionen, endlich viele Eigenschwingungen besitzt.
Die Debye-Frequenz
berechnet sich durch

\begin{align*}
  \int_0^{\omega_D} Z(\omega) \symup{d} \omega = 3 \text{N}_L \, .
\end{align*}

Die Grenzfrequenz berechnet sich somit mit der Formel

\begin{align*}
  \omega_D^3 = \frac{18 \pi^2 \text{N}_L}{L^3 \left(\frac{1}{v_l^3} + \frac{1}{v_{tr}^3} \right)} \, .
\end{align*}

Die Molwärme im Debye-Modell berechnet sich dann durch

\begin{equation}
  C_V = \frac{9 \text{R} T^3}{\theta_D^3} \int_0^{\frac{\theta_D}{T}}
  \frac{x^3 \mathrm{e}^x} {\left(\mathrm{e}^x -1 \right)^2} \symup{d} x \, .
\end{equation}

Die Abkürzungen stehen dabei für
\begin{align*}
  x &= \frac{\hbar \omega}{\text{k} T} \\
  \frac{\theta_D}{T} &= \frac{\hbar \omega_D}{\text{k} T} \, .
\end{align*}

Die Debye-Temperatur $\theta_D$ ist eine materialspezifische Eigenschaft und
durch die Kristalleigenschaften gegeben. Auch die Debye-Funktion nährt sich
bei hohen Temperaturen dem Wert $3 \text{R}$ an. Im Tieftemperaturbereich
weist die Funktion jedoch eine $T^3$ abhängig auf, wohingegen die
Einstein-Funktion einen exponentiellen Zusammenhang beinhaltet. Die
Debye-Funktion ist dort eher Zutreffend, als die Einstein-Funktion.
