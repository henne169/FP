\section{Diskussion}
In Abbildung \ref{C_V} ist zu erkennen,
dass der grobe Verlauf der empirischen Kurve der Molwärme
mit der der theoretischen Kurve übereinstimmt.
Es sind jedoch sehr große Abweichungen vorhanden.
Außerdem kann der Tabelle \ref{tab:2} entnommen werden,
dass einige empirisch bestimmten Molwärmen den Wert von $3R = 24,94$\,J/mol$\cdot$K überschreiten.
Das wirkt sich auf die empirischen Debye-Temperaturen aus (siehe Tabelle \ref{tab:3}).
Die gemittelte Debye-Temperatur $\theta_D^{\text{Empirie}} = (212,11\pm23,04)$\,K
weicht von der theoretisch bestimmten Debye-Temperatur $\theta_D^{\text{Theorie}} = 411,27$\,K
um $48,43\%$ ab.
Außerdem fällt auf, dass in der Literatur für Kupfer eine Debye-Temperatur von $\theta_D^{\text{Literatur}} \approx 345$\,K \cite{debye} gängig ist.
Diese liegt zwischen den beiden ermittelten Debye-Temperaturen.

Grund für die Abweichungen können zum einen die nicht ideale Isolation der Gefäße sein,
sodass der Wärmeaustausch nicht verhindert werden kann.
Das Dewar-Gefäß ist nicht von allen Seiten von der Druckkammer umgeben, sodass an der Aufhängung
Wärme verloren geht bzw. zugeführt wird.
Zudem ist auch kein perfektes Vakuum realisierbar, was zur Folge hat, dass restliche Moleküle in der Druckkammer Konvektion gewährleisten.
Zum anderen kann menschliches Versagen nicht ausgeschlossen werden.
Dazu gehören Rechen-, Übertragungs- und systematische Fehler.

Es muss angemerkt werden, dass bei der Messung aufmerksam Aufgepasst werden muss, damit die Zeiten bei Erhitzung der Probe rechtzeitig gestoppt werden.
Außerdem ist am Anfang für die ersten drei Messungen (siehe Tabelle \ref{tab:3}) das Voltmeter stehengeblieben, was aber zu keinen gravierenden Fehlern geführt hat.

Alles in einem stimmen die Größenordnungen der empirischen Werte mit den theoretischen Werten überein.
