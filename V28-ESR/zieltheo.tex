\section{Ziel}
Ziel des Experiments ist, das magnetische Moment freier Elektronen zu messen.

\section{Theorie}
Die Hüllenelektronen eines Atoms, die einen Bahndrehimpuls besitzen, erzeugen ein magnetisches Moment.
Freie Elektronen erzeugen auch ein magnetisches Moment, was auf ihren Eigendrehimpuls zurückzuführen ist.
Der Eigendrehimpuls wird als Spin bezeichnet.
Der Zusammenhang zwischen dem Bahndrehimpuls und magnetischem Moment ist gegeben durch
\begin{equation}
  \mu_z = -\frac{e_0}{2m_0}m\hbar := \mu_Bm,
\end{equation}
wobei $\mu_B$ das Bohrsche Magneton bezeichnet.

Die Wellenfunktion für ein Atom mit einem Außenelektron kann in einen Radialteil und Winkelteil aufgeteilt werden und lautet
\begin{equation*}
  \Psi_{n,l,m}(r,\theta,\phi) = R_{n,l}(r)\Theta_{l,m}(\theta)\Phi(\phi) = R_{n,l}\Theta_{l,m}\frac{\exp(im\phi)}{\sqrt{2\pi}}.
\end{equation*}
Hierbei ist $n$ die Hauptquantenzahl, $l$ die Bahndrehimpulsquantenzahl und $m$ die Orientierungsquantenzahl.
Ein magnetisches Moment, das in ein äußeres homogenes Magnetfeld gebracht wird, enthält die potentielle Energie
\begin{equation*}
  E_{\text{mag}}(m_l) = \mu_z\cdot{B} = m_l\cdot\mu_B\cdot{B}.
\end{equation*}
In einem Magnetfeld kommt es zu einer Aufspaltung der Energieniveaus.
Dies wird als Zeeman-Effekt bezeichnet.
Der Drehimpulsbetrag und die Richtung sind dabei gequantelt.
Die $z$-Komponente (in Magnetfeldrichtung) des Drehimpulses kann die Werte
\begin{align*}
  l_z = m_l \hbar
\end{align*}
annehmen, wobei die Orientierung die Werte
\begin{align*}
  m_l = -l, (-l+1),(-l+2),...,(l-1),l
\end{align*}
also $2l+1$ Einstellmöglichkeiten annehmen kann.
Für den Spin eines Elektrons mit Spinquantenzahl $s = \sfrac{1}{2}$ und Spin-Orientierungsquantenzahl $m_s = \pm\sfrac{1}{2}$ gibt es zwei Einstellmöglichkeiten:
\begin{equation*}
  S_z = m_s\hbar = \pm\frac{1}{2}\hbar.
\end{equation*}
Das zum Spin des Elektrons gehörende magnetische Moment lautet
\begin{equation}
  \mu_{S_z} = -gm_s\mu_B.
\end{equation}
Der Zahlenfaktor $g$ bezeichnet das gyromagnetische Verhältnis bzw. den Landé-Faktor.

\section{Beschreibung des Messprinzips}
