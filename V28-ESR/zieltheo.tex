\section{Ziel}
Ziel des Experiments ist, das magnetische Moment freier Elektronen zu messen.

\section{Theorie}
\subsection{Der Zusammenhang zwischen Bahndrehimpuls und magnetischem Moment}
Die Wellenfunktion für ein Atom mit einem Außenelektron kann in einen Radialteil und Winkelteil aufgeteilt werden und lautet
\begin{equation*}
  \Psi_{n,l,m}(r,\theta,\phi) = R_{n,l}(r)\Theta_{l,m}(\theta)\Phi(\phi) = R_{n,l}\Theta_{l,m}\frac{\exp(im\phi)}{\sqrt{2\pi}}.
\end{equation*}
Hierbei ist $n$ die Hauptquantenzahl, $l$ die Bahndrehimpulsquantenzahl und $m$ die Orientierungsquantenzahl.
Für die Funktionen $R$, $\Theta$ und $\Phi$ gelten jeweils die Normierungsbedingung:
\begin{gather*}
  \int_0^{\infty}r^2R^2dr = 1, \\
  \int_0^{\pi}\Theta^2\sin{\theta}d\theta = 1, \\
  \int_0^{2\pi}\Phi\Phi^*d\phi = 1.
\end{gather*}
Die Stromdichte eines Teilchenstromes lässt sich mit
\begin{equation*}
  \vec{S} = \frac{\hbar}{2im_0}(\psi^*\nabla\psi - \psi\nabla\psi^*)
\end{equation*}
berechnen.
Damit lässt sich auf die elektrische Stromdichte
\begin{equation*}
  j_{\phi} = -e_0S_{\phi}
\end{equation*}
schließen.
Das gesamte magnetische Moment ergibt sich zu
\begin{equation}
  \mu_z = -\frac{e_0}{2m_0}m\hbar := \mu_Bm
\end{equation}
