\section{Diskussion}
Um die Messung qualitativ bewerten zu können wird zunächst eine Übersicht
über die Messwerte, Theoriewerte und deren Abweichungen erstellt.

\begin{table}[H]
  \centering
\begin{tabular}{c|cc}
  \toprule
& $B_{\text{Erd}} \, [\SI{}{\micro\tesla}]$ & $g$\\
 \midrule
  \text{Messwert} & 43 \pm 5 & 1,9 \pm 0,1 \\
  \text{Theoriewert} & 44 \cite{erde} & 2,002 \cite{gyro}\\
  \text{Sigmaumgebung} & 1. & 2. \\
\bottomrule
\end{tabular}
\caption{Vergleich der im Versuch gemessenen Werte mit Theoriewerten}
\label{tab:vergleich}
\end{table}

Beide gemessenen Werte sind als sehr gut zu bewerten. Trotzdem gibt es einige
Fehlerquellen die berücksichtigt werden müssen. Eine Fehlerquelle kann die
Ausrichtung der Helmholtzspule zum Erdmagnetfeld darstellen. Das Abgleichen
der Brücke stellt auch eine Fehlerquelle dar, da kein absoluter Nullpunkt
erreicht werden konnte, und der Regler für den Widerstand nicht mehr fein
drehbar war. Des Weiteren konnte die Signalfrequenz nur manuell eingestellt werden,
wodurch Fehler passieren können. Bei der Suche nach dem Hauptmaximum der
Resonanzfrequenz kann es zudem passieren, dass ein Nebenmaximum und nicht das
Hauptmaximum gefunden wurde. Zudem reagierte der X-Y-Schreiber sehr
empfindlich auf äußere Einflüsse. Diese Fehlerquellen sind jedoch wahrscheinlich
eher gering, da die Messung trotzdem gute Ergebnisse geliefert hat.
