\section{Auswertung}
\subsection{Kontrastmessung}
Zunächst muss für eine bessere Qualität der späteren Messung der Kontrast
ermittelt werden. Dazu wird zunächst das Interferometer justiert und dann eine
Doppel-Glasplatte in das Interferometer eingebaut. Zwei Photodioden messen dann
 von der Intensität abhängige Spannungen. Die Spannungen werden dann von einem
Gerät in eine Differenzspannung umgewandelt und angezeigt. Dann wird der
Polarisationswinkel geändert und jeweils die maximale und minimale Differenz
ermittelt. Mit Hilfe von Formel \eqref{eqn:kontrast} wird dann der Kontrast
berechnet. Die gemessenen Spannungen mit den zugehörigen Winkeln werden in
Tabelle \ref{tab:kontrast} dargestellt.

\begin{table}
[H]
  \centering
\begin{tabular}{cccc}

  \toprule
$\phi \ [\si{\degree}]$ & $U_min \ [\SI{}{\volt}]$ &
$U_max \ [\SI{}{\volt}]$ & $K$ \\
\midrule

195 & \SI{1.28}{} & \SI{2.70}{} & \SI{0.36}{} \\

180 & \SI{1.52}{} & \SI{1.82}{} & \SI{0.09}{} \\

165 & \SI{0.76}{} & \SI{2.01}{} & \SI{0.45}{} \\

150 & \SI{0.24}{} & \SI{1.70}{} & \SI{0.75}{} \\

135 & \SI{0.06}{} & \SI{1.30}{} & \SI{0.91}{} \\

120 & \SI{0.10}{} & \SI{0.95}{} & \SI{0.81}{} \\

105 & \SI{0.26}{} & \SI{0.82}{} & \SI{0.52}{} \\

90  & \SI{0.67}{} & \SI{0.81}{} & \SI{0.09}{} \\

75  & \SI{0.47}{} & \SI{1.57}{} & \SI{0.54}{} \\

60  & \SI{0.15}{} & \SI{2.66}{} & \SI{0.89}{} \\

45  & \SI{0.14}{} & \SI{3.41}{} & \SI{0.92}{} \\

30  & \SI{0.54}{} & \SI{3.13}{} & \SI{0.71}{} \\

15  & \SI{1.21}{} & \SI{2.75}{} & \SI{0.39}{} \\

0   & \SI{1.64}{} & \SI{1.90}{} & \SI{0.07}{} \\

-15 & \SI{0.70}{} & \SI{2.00}{} & \SI{0.48}{} \\

\bottomrule
\end{tabular}

\caption{Berechneter Kontrast in Abhängigkeit der Polarisationsrichtung}
\label{tab:kontrast}
\end{table}



Nach Auftragen der Messwerte kann auf einen Zusammenhang schließen, der der
Betragsfunktion des Sinus ähnelt. Die Ausgleichsfunktion lautet somit
\begin{equation}
  f(\phi) = a \cdot \lvert \sin(b \cdot \phi +c) \rvert +d
\end{equation}
Die Messwerte und die Ausgleichsfunktion sind in Abbildung \ref{fig:kontrast}
dargestellt.

Für die Parameter ergeben sich die folgenden Werte
\begin{align*}
  a &= 
\end{align*}

\begin{figure}[H]
  \centering
  \includegraphics[width=\textwidth]{kontrast.pdf}
  \caption{Kontrast in Abhängigkeit der Polarisationsrichtung mit
  Ausgleichsrechnung}
  \label{fig:kontrast}
\end{figure}
