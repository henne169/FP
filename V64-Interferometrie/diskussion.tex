\section{Diskussion}
Im folgenden werden die Messergebnisse qualitativ bewertet und es wird auf
mögliche Fehlerquellen verwiesen.
\subsection{Kontrastmessung}
Die Kontrastmessung kann nur schlecht bewertet werden, da keine Vergleichswerte
herangezogen werden können. Die Messung diente dazu, die späteren Messungen
qualitativ zu verbessern. Zur Messung des Kontrasts wird die größte und kleinste
Spannungsdifferenz der Photodioden gesucht. Da dieses Verfahren manuell, also
nicht automatisiert, durchgeführt wird, ist es möglich, dass nicht immer der
höchste und der kleinste Intensitätswert gefunden wird. Des Weiteren können Justagefehler nicht
ausgeschlossen werden. In dem kompletten Experiment sind zudem
Wechselwirkungen mit fremden Lichtquellen möglich.

\subsection{Messung der Brechunsindizes}
Zur Übersichtlichkeit werden die berechneten Werte den Theoriewerten in einer
Tabelle \ref{tab:vergleich} gegenübergestellt.

\begin{table}[H]
  \centering
\begin{tabular}{ccc}
  \toprule
& Glasmessung & Luftmessung \\
\midrule
\text{Gemessene Indizes} & \SI{1.526 \pm 0.097}{} & {1.0001329 \pm 0.0000032}{} \\
\text{Theoriewert} & \SI{1.5}{} & \SI{1.000272}{} \\
\bottomrule
\end{tabular}
\caption{Vergleich der Theoriewerte mit den aus dem Experiment ermittelten
Werten}
\label{tab:vergleich}
\end{table}


Bei der Messung des Brechungsindex von Glas ist eine hohe Übereinstimmung mit
dem Theoriewert zu erkennen. Dieser Wert entspricht jedoch dem Mittelwert der
einzelnd berechneten Brechungsindizes. Bei den einzelnen Werten sind einige
Werte, die von dem Theoriewert abweichen. Das kann zu einem daran liegen, dass
die Glasplatte schon zu Beginn ungerade in den Aufbau eingebracht wurden. Des
Weiteren sind Störungen durch äußere Einflüsse möglich. Zudem wird die
Winkeländerung erneut manuell eingestellt, wodurch es zu Ungenauigkeiten
kommen kann.
Die Anzahl der Interferenzmaxima wird zudem mit einem
Oszilloskop gemessen. Ist bei der Justage ein Fehler aufgetreten, so ist es
möglich, dass die Differenzspannung ungenau ist. Das ist hier wahrscheinlich
jedoch weniger der Fall, da selbst einzelne, abweichende Brechungsindizes noch
in der Nähe des Theoriewerts sind. \\

Die Luftmessung hängt, wie schon in der Auswertung beschrieben, von der Lage
der Sinusfunktion beim Einlassen der Luft in die Gaszelle und von dem
Endpunkt der Messung ab. Des Weiteren
ist eine statistische Bewertung der Messwerte erst nach mehrfachen
Messreihen möglich.
