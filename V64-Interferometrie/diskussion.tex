\section{Diskussion}
Im folgenden werden die Messergebnisse qualitativ bewertet und es wird auf
mögliche Fehlerquellen verwiesen.
\subsection{Kontrastmessung}
Die Kontrastmessung kann nur schlecht bewertet werden, da keine Vergleichswerte
herangezogen werden können. Die Messung diente dazu, die späteren Messungen
qualitativ zu verbessern. Zur Messung des Kontrasts wird die größte und kleinste
Spannungsdifferenz der Photodioden gesucht. Da dieses Verfahren manuell, also
nicht automatisiert, durchgeführt wird, ist es möglich, dass nicht immer die
Absolutwerte gefunden wurden. Des Weiteren können Justagefehler nicht
ausgeschlossen werden. In dem kompletten Experiment sind zudem
Wechselwirkungen mit fremden Lichtquellen möglich.

\subsection{Messung der Brechunsindizes}
Zur Übersichtlichkeit werden die berechneten Werte den Theoriewerten in einer
Tabelle \ref{tab:vergleich} gegenübergestellt.

\begin{table}[H]
  \centering
\begin{tabular}{ccc}
  \toprule
& Glasmessung & Luftmessung \\
\midrule
\text{Gemessene Indizes} & \SI{1.526 \pm 0.097}{} & {1.0001329 \pm 0.0000032}{} \\
\text{Theoriewert} & \SI{1.5}{} & \SI{1.000272}{} \\
\bottomrule
\end{tabular}
\caption{Vergleich der Theoriewerte mit den aus dem Experiment ermittelten
Werten}
\label{tab:vergleich}
\end{table}

