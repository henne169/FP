\section{Durchführung}
\subsection{Justierung der Apparatur}
Zunächst muss der Srahlengang des Lasers mit Hilfe von sogenannten "Paddles" justiert werden,
sodass beide am ersten PBSC gestreuten Strahlen möglichst den gleichen Weg durchlaufen.
%Abbildung \ref{fig:sagnac}

\subsection{Messung der Intensität in Abhängigkeit des Polarisationswinkels}
Es muss der Kontrast für eine bessere Qualität der Messung ermittelt werden.
Dazu werden zunächst durch Verschieben des zweiten Spiegels die Teilstrahlen räumlich voneinander getrennt.
Somit laufen die Teilstrahlen parallel aneinander vorbei.
Um einen Gangunterschied zwischen den Strahlen zu erzeugen,
werden zwei transparente Glasplättchen,
die im Winkel von 20° zueinander angeordnet sind,
eingebaut, welche jeweils von einem der Teilstrahlen durchlaufen werden.
Vor dem ersten PBSC wird ein rotierbarer Polarisationsfilter eingebaut,
mit dem der Polarisationswinkel geändert werden kann.
Zwei Photodioden Messen die Intensität des Lichts in Form von Spannungen.
Es wird die Differenz der Spannungen aufgenommen.
Damit ergibt sich für die größte Spannungsdifferenz der beiden Teilstrahlen der optimale Interferenzkontrast des Interferometers.


\subsection{Messung des Brechungsindex von Glas}

\subsection{Messung des Brechungsindex von Luft}
