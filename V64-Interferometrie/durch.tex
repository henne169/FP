\section{Durchführung}
\subsection{Justierung der Apparatur}
Zunächst muss der Srahlengang des Lasers mit Hilfe von sogenannten "Paddles" justiert werden,
sodass beide am ersten PBSC gestreuten Strahlen möglichst den gleichen Weg durchlaufen.
%Abbildung \ref{fig:sagnac}

\subsection{Messung der Intensität in Abhängigkeit des Polarisationswinkels}
Es muss der Kontrast für eine bessere Qualität der Messung ermittelt werden.
Dazu werden zunächst durch Verschieben des zweiten Spiegels die Teilstrahlen räumlich voneinander getrennt.
Somit laufen die Teilstrahlen parallel aneinander vorbei.
Um einen Gangunterschied zwischen den Strahlen zu erzeugen,
werden zwei transparente Glasplättchen,
die im Winkel von 20° zueinander angeordnet sind,
eingebaut, welche jeweils von einem der Teilstrahlen durchlaufen werden.
Vor dem ersten PBSC wird ein rotierbarer Polarisationsfilter eingebaut,
mit dem der Polarisationswinkel geändert werden kann.
Zwei Photodioden Messen die Intensität des Lichts in Form von Spannungen.
Es wird die maximale und minimale Differenz der Spannungen aufgenommen.
Damit ergibt sich für die größte und kleinste Spannungsdifferenz der beiden Teilstrahlen
die höchste bzw. niedrigste Intensität.

\subsection{Messung des Brechungsindex von Glas}
Es wird der im Abschnitt \ref{sec:SI} beschriebene Aufbau verwendet.
Die im vorherigen Abschnitt erwähnten Glasplättchen werden für diese Messreihe in den Interferometer eingebaut.
Außerdem befinden sich die Plättchen auf einer rotierbaren Vorrichtung.
Durch Rotation der Glasplättchen, verändert sich die Interferenz, die auf dem Oszilloskop beobachtbar sind.
Die Interferenzmaxima werden gezählt und in Abhängigkeit des Rotationswinkels der Vorrichtung dokumentiert.
Dabei kann jeder aufsteigende Nulldurchgang auf dem Oszilloskop als Interferenzmaximum interpretiert werden.

\subsection{Messung des Brechungsindex von Luft}
Für diese Messriehe wird eine Gaskammer in den Strahlenganz eines Strahls eingebaut.
Durch eine Vakuumpumpe wird in der Kammer möglichst ein Vakuum erzeugt.
Während mit Hilfe eines Ventils nach und nach Luft in die Kammer gelassen wird,
werden mittels Oszilloskop die entstehenden Interferenzmaxima gezählt.
