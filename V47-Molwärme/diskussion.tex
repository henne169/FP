\section{Diskussion}
In Abbildung \ref{C_V} ist Erkennbar,
dass der grobe Verlauf der empirischen Kurve der Molwärme
mit der der theoretischen Kurve übereinstimmt.
Es sind jedoch sehr große Abweichungen vorhanden.
Außerdem kann der Tabelle \ref{tab:2} entnommen werden,
dass einige empirisch bestimmten Molwärmen den Wert von $3R = 24,94$J/mol$\cdot$K überschreiten.
Das wirkt sich auf die empirischen Debye-Temperaturen aus (siehe Tabelle \ref{tab:3}).
Die gemittelte Debye-Temperatur $\theta_D^{\text{Empirie}} = (212,11\pm23,04)$K
weicht von der theoretisch bestimmten Debye-Temperatur $\theta_D^{\text{Theorie}} = 411,27$K
um $48,43\%$ ab.
Außerdem fällt auf, dass in der Literatur für Kupfer eine Debye-Temperatur von $\theta_D^{\text{Literatur}} = 345$K \cite{debye} gängig ist.
