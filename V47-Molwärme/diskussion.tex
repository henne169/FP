\section{Diskussion}
Zunächst muss angemerkt werden, dass am Anfang für die ersten drei Messungen (siehe Tabelle \ref{tab:1}) das Voltmeter stehengeblieben ist.
Daher musste für die darauf folgenden Messungen ein alternatives Voltmeter verwendet werden.
In den Tabellen \ref{tab:1}, \ref{tab:2} und \ref{tab:3} sind die betroffenen Messungen rot markiert.
In Abbildung \ref{C_V} sind die betroffenen resultierenden fehlerbehafteten Größen grün markiert.
Für weitere Berechnungen wurden diese nicht berücksichtigt.

Allerdings ist zu in Abbildung \ref{C_V} erkennen,
dass der grobe Verlauf der empirischen Kurve der Molwärme
mit der der theoretischen Kurve übereinstimmt.
Es sind jedoch sehr große Abweichungen vorhanden.

Außerdem kann der Tabelle \ref{tab:2} entnommen werden,
dass einige empirisch bestimmten Molwärmen den Wert von $3R = 24,94$\,J/mol$\cdot$K überschreiten.
Das wirkt sich auf die empirischen Debye-Temperaturen aus (siehe Tabelle \ref{tab:3}).
Die gemittelte Debye-Temperatur $\theta_D^{\text{Empirie}} = (218,67\pm31,07)$\,K
weicht von der theoretisch bestimmten Debye-Temperatur $\theta_D^{\text{Theorie}} = 332,06$\,K
um $34,15\%$ ab.
Außerdem fällt auf, dass in der Literatur für Kupfer eine Debye-Temperatur von $\theta_D^{\text{Literatur}} \approx 345$\,K \cite{debye} gängig ist.
Dies liegt daran, dass bei dem Debye-Modell genähert wird, dass die Frequenz und Ausbreitungsrichtung einer Welle keinen Einfluss auf ihre Phasengeschwindigkeit hat (siehe Abschnitt \ref{debmod}).
Daher ist $\theta_D^{\text{Literatur}} > \theta_D^{\text{Theorie}}$, da in der Realität die Phasengeschwindigkeit sehr wohl davon anhängt.

Grund für die Abweichungen der empirischen Werte können zum einen die nicht ideale Isolation der Gefäße sein,
sodass der Wärmeaustausch nicht verhindert werden kann.
Der Rezipient ist nicht komplett vom Vakuum umgeben, sodass über die Aufhängung Wärme verloren geht oder zugeführt wird.


Alles in einem stimmen die Größenordnungen der empirischen Werte mit den theoretischen Werten überein.
