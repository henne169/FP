\section{Theorie}
Der Faraday-Effekt oder auch die Faraday-Rotation, beschreibt die Drehung der
Polarisationsebene eines Lichtstrahls beim Durchlaufen von Materie unter dem
Einfluss eines Magnetfeldes. Mit Hilfe des Effektes ist es möglich die
Bandstruktur in einem Halbleiter zu verstehen und die effektive Elektronenmasse
in diesem zu bestimmen.

\subsection{Das quantenmechanische Bändermodell}
