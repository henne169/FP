\section{Auswertung}
\subsection{Bestimmung der maximalen Kraftflussdichte des Magnetfeldes}

Zunächst werden in Abbildung \ref{fig:B(z)} die aufgenommenen Daten (siehe Tabelle \ref{tab:B(z)}) der Magnetfeldstärke gegen den Ort in der Spule aufgetragen.

\begin{figure}[H]
  \centering
  \includegraphics
  [width=\textwidth]{B-Feld.pdf}
  \caption{Kraftflussdichte des Magnetfeldes in Abhängigkeit von dem Ort der Spule}
  \label{fig:B(z)}
\end{figure}

\begin{table}
  \centering
  \begin{tabular}{cc}
    \toprule
    Ort $z$ / [cm] & Magnetfeldstärke $B$ / [mT] \\
    \midrule
    -002 & 407.0 \\
    -100 & 1.300 \\
    -080 & 1.880 \\
    -060 & 1.000 \\
    -040 & 0.970 \\
    -035 & 1.100 \\
    -030 & 1.690 \\
    -025 & 4.360 \\
    -020 & 16.60 \\
    -018 & 30.50 \\
    -016 & 59.70 \\
    -014 & 119.4 \\
    -012 & 225.0 \\
    -010 & 330.0 \\
    -009 & 357.0 \\
    -008 & 374.0 \\
    -007 & 386.0 \\
    -006 & 395.0 \\
    -005 & 400.0 \\
    -004 & 404.0 \\
    -003 & 406.0 \\
    -001 & 407.0 \\
    0000 & 405.0 \\
    0001 & 403.0 \\
    0002 & 398.0 \\
    0003 & 391.0 \\
    0004 & 381.0 \\
    0005 & 367.0 \\
    0006 & 345.0 \\
    0007 & 315.0 \\
    0008 & 275.0 \\
    0009 & 225.0 \\
    0010 & 173.0 \\
    0012 & 89.50 \\
    0014 & 46.10 \\
    0016 & 25.20 \\
    0018 & 14.00 \\
    0020 & 8.170 \\
    0025 & 2.630 \\
    0030 & 1.390 \\
    0035 & 1.120 \\
    0040 & 1.080 \\
    0060 & 1.140 \\
    0080 & 1.460 \\
    0100 & 1.340 \\
    \bottomrule
  \end{tabular}
  \caption{Magnetfeldstärke in Anhängigkeit von dem Ort}
  \label{tab:B(z)}
\end{table}

Die maximale Kraftflussdichte des Magnetfeldes entspricht der Magnetfeldstärke am Ort der Probe.
Somit ergibt sich die Kraftflussdichte von
\begin{align}
  B_0 = 407\si{\milli\tesla}.
  \label{b}
\end{align}

\subsection{Bestimmung der effektiven Masse}
In Tabelle \ref{tab:proben} werden die drei verwendeten Proben mit einigen Daten aufgelistet.

\begin{table}
  \centering
  \begin{tabular}{ccc}
    \toprule
    Probe & Dotierung / [$\frac{1}{\text{cm}^3}$] & Dicke $L$ / [mm] \\
    \midrule
    1. GaAs & $1.2\cdot10^{8}$ & 1.36 \\
    2. GaAs & $2.8\cdot10^{18}$ & 1.296 \\
    3. GaAs & - & 5.11 \\
    \bottomrule
  \end{tabular}
  \caption{Dotierung und Dicke der verwendeten GaAs-Proben}
  \label{tab:proben}
\end{table}

Die Rotatinswinkel der Polarisationsebenen werden mit Hilfe der Formel
\begin{equation}
  \theta = \sqrt{(\frac{1}{2}(\theta_1 - \theta_2))^2}
\end{equation}
berechnet.
Die zu den Proben entsprechenden Winkel werden in Tabelle \ref{tab:winkel} aufgelistet.

\begin{table}

  \centering
\begin{tabular}{c|c|c|c|c|c|c}

  \toprule
  & \multicolumn{3}{|c|}{Winkel $\theta_{1,2,3}$} & \multicolumn{3}{|c|}{Normierter Winkel $({\frac{\theta}{L}})_{1,2,3}$ / [$\frac{1}{\text{mm}}$]} \\
  Wellenlänge $\lambda$ / [$\si{\micro}$m] & Probe 1 & Probe 2 & Probe 3 & Probe 1 & Probe 2 & Probe 3 \\

  \midrule
  1,06 & 9,695° & 11,455° & 12,600° & 7,129° &  8,839°  & 2,466°  \\

  1,29 & 0,445° & 19,010° &  7,390° & 1,390° &  4,726°  & 5,682°  \\
  1,72 & 1,625° & 28,125° & 21,705° & 1,779° &  4,136°  & 0,442° \\
  1,96 & 2,420° &  5,360° &  2,260° & 1,195° & 21,701°  & 4,248°  \\
  2,16 & 1,890° &  6,125° & 29,035° & 4,051° &  7,299°  & 0,366° \\

  2,34 & 3,400° & 54,725° &  1,840° & 7,088° &  3,457°  & 0,141° \\
  2,51 & 9,640° &  4,480° &  0,720° & 2,500° & 42,226°  & 0,360° \\
  2,65 & 5,510° &  9,460° &  1,870° & 0,327° & 14,668°  & 1,446°  \\
  \bottomrule
\end{tabular}
\caption{Rotationswinkel der Polarisationsebenen der verwendeten Proben bei verschiedenen Wellenlängen.}

\label{tab:winkel}
\end{table}



Im Diagramm \ref{fig:0(y)} werden übersichtshalber die normierten Rotationswinkel gegenüber der Wellenlängen aufgetragen.

\begin{figure}[H]
  \centering
  \includegraphics
  [width=\textwidth]{winkelplot.pdf}
  \caption{Rotationswinkel in Abhängigkeit von der Wellenlänge der Proben im Vergleich}
  \label{fig:0(y)}
\end{figure}

Als nächstes wird der Betrag der Differenz der Faraday-Rotation zwischen dotierter und reiner Probe gebildet.
Dies kann Tabelle \ref{tab:difftab} entnommen werden.

\begin{table}

  \centering
\begin{tabular}{c|c|c}

  \toprule
Wellenlänge $\lambda$ / [$\si{\micro}$m] & $|\frac{\theta_1}{L_1}-\frac{\theta_3}{L_3}$| & |$\frac{\theta_2}{L_2}-\frac{\theta_3}{L_3}$| \\

\midrule
1,06 & 4,663 &  6,373 \\

1,29 & 4,292 &  0,956 \\

1,72 & \textcolor{red}{1,337} &  3,694 \\

1,96 & 3,053 & 17,453 \\

2,16 & 3,685 &  6,933 \\

2,34 & \textcolor{red}{6,947} &  3,316 \\

2,51 & 2,14  & \textcolor{red}{41,866} \\

2,65 & 1,119 & 13,222 \\
\bottomrule

\end{tabular}
\caption{Beträge der Differenzen der Faraday-Rotationen zwischen dotierter und reiner Probe}

\label{tab:difftab}
\end{table}



Die Formel
\begin{equation}
  \frac{\theta}{L} = \frac{e_0^3}{8\pi^2\epsilon_0c^3m^2}\frac{NB}{n}\cdot\lambda^2
\end{equation}
hat die Form
\begin{equation}
  \frac{\theta}{L} = \alpha\cdot \lambda^2.
\end{equation}
Mit einer Ausgleichsrechnung lässt sich der experimentelle Faktor $\alpha$ bestimmen.

Den Abbildungen \ref{fig:winkeldiff1} und \ref{fig:winkeldiff2} können die Differenzen zwischen dotierter und reiner Probe in Abhängigkeit von der Wellenlänge zum Quadrat und die entsprechende Ausgleichskurve entnommen werden.

\begin{figure}[H]
  \centering
  \includegraphics
  [width=\textwidth]{winkeldiffplot1.pdf}
  \caption{Ausgleichskurve für die Differenz der Faradayrotation zwischen dotierter und reiner Probe in Abhängigkeit von der Wellenlänge zum Quadrat der Probe 1.}
  \label{fig:winkeldiff1}
\end{figure}
Der mit Python berechnete Faktor ist
\begin{equation}
  \alpha_1 = 0.0885 \pm 0.043
\end{equation}

\begin{figure}[H]
\centering
\includegraphics
[width=\textwidth]{winkeldiffplot2.pdf}
\caption{Ausgleichskurve für die Differenz der Faradayrotation zwischen dotierter und reiner Probe in Abhängigkeit von der Wellenlänge zum Quadrat der Probe 2.}
\label{fig:winkeldiff2}
\end{figure}
Der mit Python berechnete Faktor ist
\begin{equation}
    \alpha_2 = 0.504 \pm 0.151
\end{equation}

Die effektive Masse lässt sich nun bestimmen, indem
\begin{equation}
  \alpha_{1,2} = \frac{e_0^3N_{1,2}B}{8\pi^2\epsilon_0c^3m^2n}
  \label{eqn:vorfaktor}
\end{equation}
nach $m$ umgestellt wird.
Es ergibt sich die Formel
\begin{equation}
  m_{1,2}^* = \sqrt{\frac{e_0^3N_{1,2}B}{8\pi^2\epsilon_0c^3n\alpha_{1,2}}}
  \label{eqn:m}
\end{equation}
für die effektive Masse.

In der folgenden Tabelle \ref{tab:var} können die nötigen Variablen und Konstanten für die Berechnung von \eqref{eqn:m} entnommen werden.
\begin{table}[H]
  \centering
  \begin{tabular}{ccccc}
    \toprule
    Bezeichnung & Formelzeichen & Wert & Einheit & Quelle \\
    \midrule
    Magnetische Flussdichte    & $B$          & $407\cdot10^{-3}$    & T                           & \eqref{b} \\
    Elementarladung            & $e_0$        & $1.602\cdot10^{-19}$ & C                           & \cite{pdg} \\
    Influenzkonstante          & $\epsilon_0$ & $8.854\cdot10^{-12}$ & $\sfrac{\text{F}}{\text{m}}$ & \cite{pdg} \\
    Vakuumlichtgeschwindigkeit & $c$          & $2.9979\cdot10^{8}$  & $\sfrac{\text{m}}{\text{s}}$               & \cite{pdg} \\
    Dotierung der Probe 1      & $N_1$        & $1.2\cdot10^{8}$     & $\sfrac{1}{\text{cm}^3}$     & \ref{tab:proben} \\
    Dotierung der Probe 2      & $N_2$        & $2.8\cdot10^{18}$    & $\sfrac{1}{\text{cm}^3}$     & \ref{tab:proben} \\
    Brechungsindex von GaAs    & $n$          & $3.57$               & -                           & \cite{cb} \\
    Elektronenmasse            & $m_e$        & $9.109\cdot10^{-31}$ & kg                          & \cite{pdg} \\
    \bottomrule
  \end{tabular}
  \caption{Einige Variablen und Konstanten}
  \label{tab:var}
\end{table}

Unter Berücksichtigung der Gaußschen Fehlerfortpflanzung ergeben sich
\begin{gather*}
  m_1^* = (5.8 \pm 1.4)\cdot10^{-33}\text{kg} = (0.0064 \pm 0.002)m_e \\
  m_2^* = (3.7 \pm 0.6)\cdot10^{-28}\text{kg} = (4.1 \pm 0.6)m_e
\end{gather*}

Der Literaturwert der effektiven Masse des Elektrons im tiefsten Minimum des Leitungsbandes \cite{ru} beträgt
\begin{equation*}
  m_{\Gamma} = 0.063 m_e
\end{equation*}
