\section{Diskussion}

Die Kurve des Magnetfeldes in Abbildung \ref{fig:B(z)} weist einen realistischen Verlauf auf.
Das Maximum kann hier gut abgeschätzt werden. \\

In Abb. \ref{fig:0(y)} sind keine Regelmäßigkeiten der Rotationswinkel in Abhängigkeit der Wellenlägen zu sehen.
Dies ist ein Indiz einer fehlgeschlagenen Messung. \\

In den Abbildungen \ref{fig:winkeldiff1} und \ref{fig:winkeldiff2} sind große Abweichungen vorzuweisen.
Die Abweichungen der Abbildung \ref{fig:winkeldiff1} sind wesentlich größer als die der \ref{fig:winkeldiff2}.
Auch hier wird die Ungenauigkeit der Messung verdeutlicht. \\

Die empirische effektive Masse $m_1^*$ weicht um 96\% von der theoretischen effektiven Masse $m_{\Gamma}$ ab.
Die effektive Masse $m_2^*$ weicht um 99,9\% vom Theoriewert ab. \\

Ein Grund für die Abweichungen kann der fast immer nicht mögliche Abgleich der Spannung am Oszilloskop sein.
Meistens musste ein Minimum der Spannung abgeschätzt werden, was aber nicht immer möglich war.
Teilweise hat sich die Spannung garnicht verändert, sodass die Abschätzung des Minimums erschwert wurde.
Eine mögliche Fehlerquelle kann eine fehlgeschlagene Justierung der Apparatur sein.
Durch die falsche Justierung können äußere Einflüsse, wie Licht etc. für falsche Werte gesorgt haben.
Außerdem gab es bei dem Abgleich jeweils zwei verschiedene Winkel des Goniometers, die eingestellt werden konnten,
die im Oszilloskop ein Minimum aufgewiesen haben. Dadurch wusste man nicht, welcher Winkel der richtige ist.
Dadurch können möglicherweise falsche Differenzen der Formel \eqref{eqn:thetadiff} entstanden sein.
