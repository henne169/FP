\documentclass[
  bibliography=totoc,     % Literatur im Inhaltsverzeichnis
  captions=tableheading,  % Tabellenüberschriften
  titlepage=firstiscover, % Titelseite ist Deckblatt
]{scrartcl}
\usepackage{scrhack}
\usepackage[aux]{rerunfilecheck}
\usepackage{amsmath}
\usepackage{amssymb}
\usepackage{mathtools}
\usepackage{longtable}
\usepackage{fontspec}
\recalctypearea % Wenn man andere Schriftarten gesetzt hat,
% sollte man das Seiten-Layout neu berechnen lassen
\usepackage{polyglossia}
\setmainlanguage{german}
\usepackage[
  math-style=ISO,    % ┐
  bold-style=ISO,    % │
  sans-style=italic, % │ ISO-Standard folgen
  nabla=upright,     % │
  partial=upright,   % ┘
  warnings-off={           % ┐
    mathtools-colon,       % │ unnötige Warnungen ausschalten
    mathtools-overbracket, % │
  },                       % ┘
]{unicode-math}
\setmathfont{Latin Modern Math}
\setmathfont{XITS Math}[range={scr, bfscr}]
\setmathfont{XITS Math}[range={cal, bfcal}, StylisticSet=1]
\usepackage[
  locale=DE,                   % deutsche Einstellungen
  separate-uncertainty=true,   % immer Fehler mit \pm
  per-mode=symbol-or-fraction, % / in inline math, fraction in display math
]{siunitx}
\usepackage[
  version=4,
  math-greek=default, % ┐ mit unicode-math zusammenarbeiten
  text-greek=default, % ┘
]{mhchem}
\usepackage[autostyle]{csquotes}
\usepackage{xfrac}
\usepackage{float}
\floatplacement{figure}{htbp}
\floatplacement{table}{htbp}
\usepackage[
  section, % Floats innerhalb der Section halten
  below,   % unterhalb der Section aber auf der selben Seite ist ok
]{placeins}
\usepackage{pdflscape}
\usepackage[
  labelfont=bf,        % Tabelle x: Abbildung y: ist jetzt fett
  font=small,          % Schrift etwas kleiner als Dokument
  width=0.9\textwidth, % maximale Breite einer Caption schmaler
]{caption}
\usepackage{subcaption}
\usepackage{graphicx}
\usepackage{wrapfig}
\usepackage{grffile}
\usepackage{booktabs}
\usepackage{microtype}
\usepackage[
  backend=biber,
  urldate=iso8601,
  date=iso8601,
  style=chem-rsc,
  articletitle=true,
]{biblatex}
\addbibresource{lit.bib}
\usepackage[
  unicode,        % Unicode in PDF-Attributen erlauben
  pdfusetitle,    % Titel, Autoren und Datum als PDF-Attribute
  pdfcreator={},  % ┐ PDF-Attribute säubern
  pdfproducer={}, % ┘
]{hyperref}
\usepackage{bookmark}
\usepackage{tikz}
\usepackage{tikz-feynman}
\usepackage[shortcuts]{extdash}

\usepackage[autostyle]{csquotes}
\usepackage{float}
\floatplacement{figure}{htbp}
\floatplacement{table}{htbp}
\usepackage[style=numeric]{biblatex}
%\addbibresource{lit.bib}
\usepackage[locale=DE,
  separate-uncertainty=true,
  per-mode=symbol-or-fraction,
  version-1-compatibility,
  decimalsymbol=comma
]{siunitx}
\usepackage{xcolor}
\usepackage{graphicx}
\usepackage{grffile}
\usepackage{longtable}
\usepackage{arydshln}
\usepackage{subcaption}
\usepackage{xfrac}
\usepackage{wrapfig}
\usepackage{microtype}

\usepackage{tikz}
\usepackage{tikz-feynman}

\usepackage{pdfpages}
\usepackage{hyperref}

\setlength\parindent{0pt}

\usepackage{pdflscape}
%\pagestyle{empty} %keine Seitennummerierung
\usepackage{a4wide}
\usepackage{pdfpages}



\author{%
  Nicole Schulte%
  \texorpdfstring{%
    \\%
    \href{mailto:nicole.schulte@udo.edu}{nicole.schulte@udo.edu}
  }{}%
  \texorpdfstring{\and}{, }%
  Hendrik Bökenkamp%
  \texorpdfstring{%
    \\%
    \href{mailto:hendrik.boekenkamp@udo.edu}{hendrik.bökenkamp@udo.edu}
  }{}%
}
\publishers{TU Dortmund – Fakultät Physik}
\setlength\parindent{0pt}
