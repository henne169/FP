\section{Ziel}
Ziel des Versuches ist die Untersuchung des Rutherfordschen Streuexperimentes,
also die Streuung von Alpha-Teilchen an einer Goldfolie.

\section{Theorie}
\section{Alphastrahlung}
Der Begriff der Strahlung wird in Teilchen- und Wellenstrahlung unterteilt. Dabei
wird die Alphastrahlung der Teilchenstrahlung zugeordnet, da der zerfallende
Kern ein Helium-4-Atomkern absondert. Dieser Prozess ist durch den quantenmechanischen
Tunneleffekt zu erklären. Alphateilchen sind positiv geladen und verfügen über
eine schwere Masse, die die Abschirmung vereinfacht.

\subsection{Wechselwirkung mit Materie}
Durchläuft ein Alphateilchen eine Materieschicht so erfährt es durch
Wechselwirkung mit den Hüllenelektron der Atome einen Energieverlust. Dieser
Energieverlust kommt durch Ionisation oder Anregung auf Grund von inelastischen
Stößen zustande. Die Flugrichtung der Alphateilchen erfährt daruch keine
Änderung. Die Bethe-Bloch Gleichung \eqref{eqn:bloch} beschreibt dabei den
Energieverlust pro Wegsträcke des Alphateilchens.

\begin{equation}
- \frac{\symup{d} E}{\symup{d} x} = \frac{4 \pi \symup{e}^4 z^2 \text{N} \text{Z}}
{\text{m_0} v^2 (4 \pi \epsilon_0)^2} \ln\frac{2 \text{m_0} v^2}{\text{I}}
\label{eqn:bloch}
\end{equation}
