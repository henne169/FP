\section{Durchführung}
In diesem Versuch werden drei Messungen durchgeführt.
Zu Beginn wird mit Hilfe eines Oszilloskops eine Energieverlustmessung beim
senkrechten Durchgang durch die Folie durchgeführt. Dafür werden zwei Messungen
benötigt. Dazu werden die Pulshöhen der Detektorimpulse, die vorher mit Hilfe
eines Amplifiers verstärkt werden, als Funktion des Kammerdrucks einmal mit und
einmal ohne Streufolie gemessen. Zum Evakuieren der Streukammer wird eine
Drehschieberpumpe verwendet. Dadurch kann die Reichweite der Alphateilchen und
die Foliendicke bestimmt werden. Für eine genauere Messung wird am Oszilloskop
das Nachleuten der Signale eingestellt. \\
Der zweite Teil des Versuches ist eine Streuwinkelmessung. Dazu wird bei
evakuiertem Behälter der Winkel des Detektors von $0$ bis \SI{20}{\degree}
geändert und die Zählrate gemessen. Dadurch kann der differentielle
Wirkungsquerschnitt berechnet werden. \\
Im letzten Teil des Experimentes werden Folien aus Bismut und Aluminium und eine
weitere Goldfolie ausgemessen. Diese Messung finden bei einem festen Winkel statt.
Bei einem großen Winkel werden die anderen beiden Materialen ausgemessen und die
Zählrate aufgenommen. Bei der zweiten Goldfolie wird ein kleinerer Winkel
gewählt und auch die Zählrate aufgenommen. Indem die beiden Goldfolien miteinander
verglichen werden, kann der Einfluss der Mehrfachstreuung ermittelt werden. Der
Vergleich mit den anderen Materialien liefert Informationen über die
Ordnungszahlabhängigkeit der Rutherford Streuung.
