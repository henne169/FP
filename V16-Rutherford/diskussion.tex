\section{Diskussion}
Vergleicht man die Abbildungen \ref{fig:2} und \ref{fig:1} in Kapitel \ref{VPO} miteinander,
ist erkennbar, dass die Verstärkung durch den Amplifier die Amplitude des Pulses
kontinuierlich macht.
Die Größenordnungen bleiben jedoch erhalten.

Bei der Bestimmung der Goldfolie (Kapitel \ref{gofodi}) weicht die berechnete Dicke um 22,5\% ab.
Dem Termschema (Abbildung \ref{fig:amnp}) kann entnommen werden, dass $\alpha$-Strahlung nicht monoenergetisch ist.
Dieses wird in der Rechnung jedoch angenommen.
Die Ausrichtung mit Augenmaß und etwaige Wölbungen, Biegungen oder sonstige Verformungen sind zudem auch mögliche Fehlerquellen.
Außerdem muss angemerkt werden, dass ein Mittelwert der fluktuierenden Pulse auf dem Oszilloskop abgeschätzt werden musste, um ein Messwert aufzunehmen.

Bei der Bestimmung des Wirkungsquerschnitts (Kapitel \ref{kap:WQ}) ist der Verlauf der empirisch gewonnenen Wirkungsquerschnitte verschieden von den Theoriewerten (siehe Abb. \ref{fig:WQ}).
Ab einem Winkel von ca. 5° nähert sich auch die empirische Kurve dem Grenzwert 0.
Es ist besonders auffällig, dass für kleine Winkel die Abweichungen riesig sind.
Dies liegt möglicherweise an der theoretischen Streuformel für kleine Winkel, dass diese gegen unendlich geht.
Außerdem wird wieder von monoenergetischen Teilchen ausgegangen.
Anschließend wird die Mehrfachstreuung vernachlässigt, dessen Einfluss in Kapitel \ref{kap:mfs} gezeigt wird.
Es ist kein linearer Zusammenhang zwischen Foliendicke und Streuung zu erkennen, was den Einfluss bestätigt.

In Kapitel \ref{5.5} stimmen innerhalb der Fehlertoleranz die empirischen Werte mit den theoretischen Werten übereinstimmen (siehe Abb. \ref{fig:Z-WQ}).
