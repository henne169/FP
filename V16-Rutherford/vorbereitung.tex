\section{Bestimmung des Bremsvermögens von Alphateilchen in Luft}
Die Reichweite von Luft kann mit der Formel \eqref{eqn:bloch} bestimmt werden.
Im Folgenden wird angenommen, dass die Luft aus 80\% Stickstoff und 20\% Sauerstoff besteht.
Die Geschwindigkeit des $\alpha$-Teilchens beträgt $v_{\alpha}= 1,625\cdot10^{7}\,\frac{\text{m}}{\text{s}}$ (siehe Abschnitt \ref{gofodi}).
Die Teilchendichte lässt sich über den Zusammenhang
\begin{align*}
  n = \rho/m
\end{align*}
berechnen.
Dabei ist die Luftdichte $\rho_{\text{Luft}} = 1,204\,\frac{\text{kg}}{\text{m}^3}$ \cite{chem2} und die gemittelte Masse
$m = 14,4 \,\text{u}$ \cite{PSE}.
Damit ist die Teilchendichte von Luft $n_{\text{Luft}} = 2,4\cdot10^{-26}\,\text{m}^{-3}$.
Die mittlere Ionisierungsenergie von Luft ist
$\overline{I}_{\text{Luft}} = 14,32\,\text{eV}$ \cite{PSE}.
Die Kernladungszahl der $\alpha$-Teilchen ist $z=2$ und die effektive Kernladungszahl von Luft beträgt
$Z_{\text{eff}} = 7,2$.
Mit diesen Daten lässt sich das Bremsvermögen berechnen. Dieses beträgt
\begin{equation*}
  -\frac{dE}{dx} =
\end{equation*}
