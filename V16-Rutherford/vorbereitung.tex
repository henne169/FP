\section{Bestimmung des Bremsvermögens von Alphateilchen in Luft}
Die Reichweite von Luft kann mit der Formel \eqref{eqn:bloch} bestimmt werden.
Im Folgenden wird angenommen, dass die Luft aus $80\,\%$ Stickstoff und $20\,\%$ Sauerstoff besteht.
Die Geschwindigkeit des $\alpha$-Teilchens beträgt $v_{\alpha}= 1,625\cdot10^{7}\,\frac{\text{m}}{\text{s}}$ (siehe Abschnitt \ref{gofodi}).
Die Teilchendichte lässt sich über den Zusammenhang
\begin{align*}
  n = \rho/m
\end{align*}
berechnen.
Dabei ist die Luftdichte $\rho_{\text{Luft}} = 1,204\,\frac{\text{kg}}{\text{m}^3}$ \cite{chem2} und die gemittelte Masse
$m = 14,4 \,\text{u}$ \cite{PSE}.
Damit ist die Teilchendichte von Luft $n_{\text{Luft}} = 5,04\cdot10^{25}\,\text{m}^{-3}$.
Die mittlere Ionisierungsenergie von Luft ist
$\overline{I}_{\text{Luft}} = 14,32\,\text{eV}$ \cite{PSE}.
Die Kernladungszahl der $\alpha$-Teilchen ist $z=2$ und die effektive Kernladungszahl von Luft beträgt
$Z_{\text{eff}} = 7,2$.
Mit diesen Daten lässt sich das Bremsvermögen berechnen. Dieses beträgt
\begin{equation*}
  -\frac{dE}{dx} = 134,66\,\frac{\text{keV}}{\text{mm}}
\end{equation*}
Wird das ideale Gasgesetz
\begin{equation}
  n = \frac{N}{V} = \frac{p}{k_{\text{B}}T}
\end{equation}
in Formel \eqref{eqn:bloch} eingesetzt,
ist ein linearer Zusammenhang zu erkennen.
Unter Berücksichtigung von Standardbedingungen ($T= 298,15\,\text{K}$), kann darüber die Druckabhängigkeit des Energieverlustes bestimmt werden.
Sobald sich die Energie des $\alpha$-Teilchens in $10^{-2}\,\text{m}$ um $5\,\%$ verringert, macht sich der Energieverlust bemerkbar.
Das heißt, dass
\begin{align*}
  -\frac{dE}{dx} = 0,05\cdot\frac{E_{\alpha}}{1\,\text{mm}}.
\end{align*}
Der Wert von $E_{\alpha}$ kann dem Abschnitt \ref{gofodi} entnommen werden.
Durch Umstellen nach $p$ kann der Druck, bei dem sich Energieverluste bemerkbar machen, berechnet werden.
Dieser beträgt
\begin{align*}
  p = 0,4\,\text{bar}.
\end{align*}
