\section{Auswertung}
\subsection{Molwärme bei konstantem Druck}
Mit Hilfe der Formel
\begin{equation}
  C_p = \frac{E \cdot M}{\Delta T \cdot m}
\end{equation}
lässt sich die Molwärme unter konstantem Druck berechnen.
Es ist $E$ die zugeführte Energie, $M$ die molare Masse, $\Delta T$ die Temperaturänderung und $m$ die Masse der Probe.

Bei dem zu untersuchenden Material handelt es sich um Kupfer.
Die entsprechenden Werte für die molare Masse und die Masse der Kupferprobe sind
\begin{gather*}
  M_{\text{Cu}} = \textcolor{red}{??}\frac{\text{g}}{\text{mol}} \\%\cite{}, \\
  m_{\text{Cu}} = 342\si{\gram}.
\end{gather*}
