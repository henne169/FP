\section{Auswertung}
\subsection{Molwärme bei konstantem Druck}
Mit Hilfe der Formel
\begin{equation}
  C_p = \frac{E \cdot M}{\Delta T \cdot m}
\end{equation}
lässt sich die Molwärme unter konstantem Druck berechnen.
Es ist $E$ die zugeführte Energie, $M$ die molare Masse, $\Delta T$ die Temperaturänderung und $m$ die Masse der Probe.

Bei dem zu untersuchenden Material handelt es sich um Kupfer.
Die entsprechenden Werte für die molare Masse und die Masse der Kupferprobe sind
\begin{gather*}
  M_{\text{Cu}} = \textcolor{red}{??}\frac{\text{g}}{\text{mol}} \\%\cite{}, \\
  m_{\text{Cu}} = 342\si{\gram}.
\end{gather*}
Die Energie lässt sich durch
\begin{equation}
  E = U\cdot I\cdot \Delta t
\end{equation}
berechnen, wobei $U$ die aufgenommene Spannung, $I$ der gemessene Strom und $\Delta t$ das Intervall ist.

Aus der Formel
\begin{equation}
  T = 0,00134R^2 + 2,296R - 243,02
\end{equation}
folgt nach der pq-Formel für die Widerstände die Formel
\begin{equation}
  R_+ = -\frac{2,296}{2\cdot0,00134} + \sqrt{ \left(\frac{2,296}{2\cdot0,00134}\right)^2 + \frac{243,02 + T}{0,00134} },
  \end{equation}
mit der sich die Widerstände des Thermoelements zur entsprechenden Temperatur berechnen lassen.
Der Tabelle \ref{tab:1} können die gemessenen Widerstände, Temperaturänderungen, Zeiten, Ströme, Spannungen, Energien und Molwärmen entnommen werden.


\begin{table}
[H]
\begin{tabular}{ccccccc}

  \toprule
$R$ [$\si{\ohm}$] & $\Delta{T}\pm\Gamma_{\Delta{T}}$ [°] & $t \pm \Gamma_t$ [s] & $I \pm \Gamma_I$ [mA] & $U \pm \Gamma_U$ [V] & $E \pm \Gamma_E$ [J] & $C_p \pm \Gamma_{C_p} $ [J/mol K]\\

\midrule
$21,1$ & $-            $ & $ -       $ & $0            $ & $0             $ & $-$            & $-$              \\

\textcolor{red}{$25,7$} & \textcolor{red}{$10,82 \pm 0,1$} & \textcolor{red}{$523 \pm 5$} & \textcolor{red}{$133,8 \pm 0,1$}
& \textcolor{red}{$13,80 \pm 0,01$} & \textcolor{red}{$ 966 \pm  9$}  & \textcolor{red}{$16,59 \pm 0,22$} \\

\textcolor{red}{$29,9$} & \textcolor{red}{$10,00 \pm 0,1$} & \textcolor{red}{$572 \pm 5$} & \textcolor{red}{$143,7 \pm 0,1$}
& \textcolor{red}{$13,88 \pm 0,01$} & \textcolor{red}{$1141 \pm 10$}  & \textcolor{red}{$21,20 \pm 0,28$} \\
\textcolor{red}{$34,1$} & \textcolor{red}{$10,00 \pm 0,1$} & \textcolor{red}{$610 \pm 5$} & \textcolor{red}{$145,8 \pm 0,1$}
& \textcolor{red}{$13,88 \pm 0,01$} & \textcolor{red}{$1234 \pm 10$}  & \textcolor{red}{$22,93 \pm 0,30$} \\
$38,3$ & $10,00 \pm 0,1$ & $577 \pm 5$ & $153,7 \pm 0,1$ & $16,20 \pm 0,01$ & $1437 \pm 13$  & $26,70 \pm 0,40$ \\

$42,4$ & $10,00 \pm 0,1$ & $378 \pm 5$ & $178,0 \pm 0,1$ & $18,75 \pm 0,01$ & $1262 \pm 17$  & $23,40 \pm 0,40$ \\

$46,6$ & $10,00 \pm 0,1$ & $365 \pm 5$ & $178,0 \pm 0,1$ & $18,75 \pm 0,01$ & $1218 \pm 17$  & $22,60 \pm 0,40$ \\

$50,7$ & $10,00 \pm 0,1$ & $327 \pm 5$ & $180,3 \pm 0,1$ & $19,03 \pm 0,01$ & $1122 \pm 17$  & $20,80 \pm 0,40$ \\

$54,8$ & $10,00 \pm 0,1$ & $315 \pm 5$ & $180,5 \pm 0,1$ & $19,60 \pm 0,01$ & $1114 \pm 18$  & $20,70 \pm 0,40$ \\

$58,9$ & $10,00 \pm 0,1$ & $375 \pm 5$ & $180,7 \pm 0,1$ & $19,08 \pm 0,01$ & $1293 \pm 17$  & $24,00 \pm 0,40$ \\

$63,0$ & $10,00 \pm 0,1$ & $376 \pm 5$ & $180,8 \pm 0,1$ & $19,10 \pm 0,01$ & $1298 \pm 17$  & $24,10 \pm 0,40$ \\

$67,0$ & $10,00 \pm 0,1$ & $384 \pm 5$ & $180,9 \pm 0,1$ & $19,12 \pm 0,01$ & $1328 \pm 17$  & $24,70 \pm 0,40$ \\

$71,0$ & $10,00 \pm 0,1$ & $406 \pm 5$ & $181,0 \pm 0,1$ & $19,13 \pm 0,01$ & $1406 \pm 17$  & $26,10 \pm 0,40$ \\

$75,1$ & $10,00 \pm 0,1$ & $397 \pm 5$ & $181,1 \pm 0,1$ & $19,10 \pm 0,01$ & $1373 \pm 17$  & $25,50 \pm 0,40$ \\

$79,0$ & $10,00 \pm 0,1$ & $361 \pm 5$ & $181,2 \pm 0,1$ & $19,14 \pm 0,01$ & $1252 \pm 17$  & $23,30 \pm 0,40$ \\

$83,0$ & $10,00 \pm 0,1$ & $346 \pm 5$ & $181,2 \pm 0,1$ & $19,14 \pm 0,01$ & $1200 \pm 17$  & $22,30 \pm 0,40$ \\

$87,0$ & $10,00 \pm 0,1$ & $299 \pm 5$ & $181,2 \pm 0,1$ & $19,14 \pm 0,01$ & $1037 \pm 17$  & $19,30 \pm 0,40$ \\

$90,9$ & $10,00 \pm 0,1$ & $275 \pm 5$ & $181,3 \pm 0,1$ & $19,15 \pm 0,01$ & $ 955 \pm 17$  & $17,70 \pm 0,40$ \\

$94,9$ & $10,00 \pm 0,1$ & $328 \pm 5$ & $181,3 \pm 0,1$ & $19,14 \pm 0,01$ & $1138 \pm 17$  & $21,10 \pm 0,40$ \\

$98,8$ & $10,00 \pm 0,1$ & $383 \pm 5$ & $181,4 \pm 0,1$ & $19,14 \pm 0,01$ & $1330 \pm 17$  & $24,70 \pm 0,40$ \\

$102,7$ & $10,00 \pm 0,1$ & $411 \pm 5$ & $181,4 \pm 0,1$ & $19,13 \pm 0,01$ & $1426 \pm 17$ & $26,50 \pm 0,40$ \\

$106,6$ & $10,00 \pm 0,1$ & $358 \pm 5$ & $181,4 \pm 0,1$ & $19,13 \pm 0,01$ & $1242 \pm 17$ & $23,10 \pm 0,40$ \\

$110,4$ & $10,00 \pm 0,1$ & $361 \pm 5$ & $181,5 \pm 0,1$ & $19,12 \pm 0,01$ & $1253 \pm 17$ & $23,30 \pm 0,40$ \\

\bottomrule
\end{tabular}

\caption{Überblick über die gemessenen und berechneten Daten.}
\label{tab:1}
\end{table}


